\documentclass[11pt]{article}
\usepackage{fullpage}

\begin{document}
\pagestyle{empty}

\section*{Facilities, Equipment, Other}

The work described in the UC Berkeley portion of this collaborative proposal will take place at two locations: 
1) U. California, Berkeley, 
2) NRAO Green Bank, WV.

\subsection*{\it University of California, Berkeley}

The UC Berkeley Radio Astronomy Laboratory (RAL) is located adjacent to the UC
Berkeley Astronomy Department. It provides access to digital and
radio-frequency test equipment necessary for the detailed characterization and
performance testing of components of the antenna feed, the analog system, and
the digital correlator.  Such items include network analyzers (HP8753, HP8754,
E5230A), frequency synthesizers covering all frequencies up to 26 GHz (e.g.
Agilent E8247), including multiple at lower bands, digitizing oscilloscopes
(e.g. HP54502 and Lecroy 9360), power meters (e.g. Agilent E4418b), spectrum
analyzers (e.g. Agilent E4407b), noise generators, RF filters, attenuators,
amplifiers, soldering stations, machine tools, and other miscellaneous
electronic and general laboratory equipment.  

RAL hosts the Simulink/Xilinx System Generator programming
environment targeting the Field-Programmable Gate Array processors on which the
proposed correlator work is based, and
also provides access to several computer-aided design software packages. 
AutoCAD Inventor is used for three-dimensional mechanical drawings. 
Optotek MMICAD is used for circuit simulation, Orcad Capture and Altium for circuit design, and
Ansys HFSS and Mentor Graphics IE3D are used for electromagnetic simulations. 

The RAL has a multiple large laboratory areas on campus that will host the UC
Berkeley development activities.  It also provides use of an off-site
observatory (Leuschner) near the campus, which will host of one of the three
small outreach/education arrays budgeted in this proposal.  The RAL collaborates with
the Berkeley Wireless Research Center and the Space Sciences Laboratory as well
as other groups around the Berkeley campus who have additional high-end digital
test equipment and software for digital signal processing.  These facilities are
all made available to members of the UC Berkeley Astronomy Department, the RAL,
and the Collaboration for Astronomy Signal Processing and Electronics Research
(CASPER).

\subsection*{\it NRAO Green Bank, West Virginia} 

The 500 sq. ft. Galford Meadow field station, located on the NRAO Green Bank, WV 
Observatory site, will be made available to this project, in coordination with
on-going PAPER activities at the same location. It houses a work area and 
sufficient space to deploy additional BAOBAB instrumentation, and includes
power, cooling, and internet access.
This site is positioned within the National Radio Quiet Zone, and is an ideal venue
for very sensitive radio frequency measurements due to its remoteness from
large urban areas and its exceptional laboratory infrastructure. A 
large amount of outdoor space is available adjacent to the station (amid 
the PAPER antenna array) for the
deployment of BAOBAB antennas. 
Electromagnetic enclosures are
used to house all instrumentation to prevent self-interference as well as
interference with Observatory telescopes, in strict adherence to the radio
emissions control policy of the Observatory's Interference Protection Group
(IPG).

In addition to the field station, NRAO will provide access to the Jansky
laboratory facilities for work space during deployments, and use of the on-site
acechoic chamber and test equipment for testing all deployed electronics for
compliance with RFI regulations.  BAOBAB will passively benefit from the routine
site maintenance that is already provided in association with the PAPER
experiment.

\end{document}
