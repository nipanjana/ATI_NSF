\documentclass[11pt]{article}
\usepackage{fullpage}
\usepackage[top=1in, bottom=1in, left=1in, right=1in]{geometry}

\begin{document}
\pagestyle{empty}

\section*{\small COLLABORATIVE RESEARCH: The BAO Broadband And Broad-beam (BAOBAB) Array}

We propose to develop a dedicated instrument for characterizing baryon acoustic
oscillations (BAO) via intensity mapping of HI 21cm emission in the redshift
range $z=0.8$--$2.5$ (400--800 MHz).  BAO features in large-scale matter
distribution have recently drawn attention as a standard ruler by which the
expansion history of the universe can be directly measured.  Measuring the BAO
wiggles at several redshifts yields measurements of 
the Hubble parameter, $H(z)$, and the angular diameter distance, $d_A(z)$, that
constrain properties of the dark energy that dominates the cosmic energy
content at $z=0$, and is the current leading theory for the accelerating
expansion of the universe.

Rather than target individual objects, 21cm intensity mapping 
measures fluctuations in neutral hydrogen emission on large scales, with two
dimensions corresponding to angles on the sky, and the third line-of-sight
dimension arising from the differential redshifting of 21cm line emission as a
function of distance. After reionization, the power spectrum of 21cm
fluctuations is expected to be a biased tracer of the matter power-spectrum,
since the remaining neutral hydrogen resides in high-density, self-shielded
regions in galaxies and other collapsed halos.  As a result, 21cm
intensity mapping presents a promising complement to spectroscopic
galaxy surveys for BAO science;
a 21cm intensity mapping experiment can probe redshifts $0.5<z<2.5$ with roughly
uniform sensitivity, without complications arising from sky emission lines in
the optical/near-infrared.

Many of the instrumental approaches,
foreground mitigation strategies, and data analysis techniques 
developed for studying reionization with the 21cm line at $z\!\sim\!10$ apply directly to the
study of BAO at $z\!\sim\!1$.  A new BAO instrument, drawing on the
considerable investments in low-frequency radio astronomy made in the past
decade, can inexpensively leapfrog existing efforts to become a global leader
in this area.  In this vein, we propose to develop the BAO Broadband and
Broad-beam (BAOBAB) array by leveraging the techniques, instrumentation, and
infrastructure of the Precision Array to Probe the Epoch of
Reionization (PAPER) and the Murchison Widefield Array (MWA).  BAOBAB will
begin in the first year with the deployment of a 25-tile array at the
NRAO site near Green Bank, WV, based on an
existing tile design using a scaled version of the PAPER sleeved dipole,
with associated analog electronics.  In the second year, BAOBAB's front-end amplifier
will be improved to reduce the receiver noise that
dominates the system temperature.  Existing tiles will be retrofitted, and an
additional 24 tiles will be deployed, for a total of 49.  BAOBAB's correlator will combine
decommissioned FPGA hardware from PAPER with new commercial GPU processors,
following the scalable FPGA/GPU correlator design pioneered by PAPER.

The primary goals of these activities
will be: 1) to develop an analog system that meets the specifications
for foreground removal, sensitivity, and scalability 
to larger array sizes, 2) to characterize foregrounds to the 21cm BAO power spectrum,
and 3) to obtain a $\ge5\sigma$ measurement of the 21cm power 
spectrum from which the neutral hydrogen fraction
at $z\sim1$ can be derived.  These accomplishments herald next-generation efforts to 
measure BAO features versus redshift with an array four-times the size.
The broader impacts of this research include the development of technology and software
shared with other astronomy efforts, the installation of
small, 8-element interferometers at each of the three host institutions for education and
outreach initiatives, and the establishment of a
collaboration spanning multiple 21cm cosmology groups that promotes
the HERA roadmap that was favored in the A2010 Decadal Survey.
The design of BAOBAB's wide-bandwidth digital correlator 
will be shared with external research groups in the US, South Africa, Australia, and
India via the CASPER collaboration.  Analysis software and techniques developed for BAOBAB will
be shared with PAPER and MWA.
%A detection of the 21cm power spectrum by BAOBAB in a
%frequency band with lower noise and foreground contamination relative to the desired signal
%may also provide an important validation of the analysis approaches for
%detecting reionization.
BAOBAB will have a strong educational emphasis, with
fundamental contributions being made by graduate students 
in instrumentation, software development and scientific analysis.
These efforts will be valuable for training the next generation of
scientists who will lead the next stages of 21cm BAO and reionization experiments.

\end{document}
